\section{Purpose}
The purpose of this document is to give a more detailed and technical overview about PowerEnJoy system. \\
This document is addressed to developers and aims to identify:
\begin{itemize}
\item The high level architecture;
\item The design patterns;
\item The main components and their interfaces provided one for another;
\item The Runtime behavior.
\end{itemize}

\section{Scope}
This document will present different level views in order to describe clearly the architecture of PowerEnJoy System. In particular we will present the component view, both high and low level, the deployment view, the runtime view and a further description of user interface, analysed in its runtime flow.\\
The System is based on two mobile applications and a web application. \\
As already explained in the RASD, the targets for the Service are: 
\begin{itemize}
\item The Drivers;
\item The Workers;
\item The Admins.
\end{itemize}
The System allows the Drivers to reserve cars from both the mobile application and the web portal. At the same time it manages the the status of the cars and, if necessary, assigns tasks to the Workers via an ad-hoc mobile application. The Admins of the System can interact with the web portal to make privileged actions.

\section{Definitions, Acronyms and Abbreviations}
\begin{description}
\item [Driver] Client of the system, i.e. the one that uses the service, reserves and drives. Every Driver must provide personal informations (name, surname, email, birthdate, a valid license and payment information.
\item [Ride] A single usage of the service, starts when the user turn on the engine, stops when the car is locked inside a Safe Area;
\item [Blocked driver] A driver that has an invalid license or invalid payment coordinates
\item [Worker] Employee of the company, perform physical actions (moving, charging etc) on cars
\item [Report] Car issue reported by the Driver during a Drive. Will later be assigned to a Worker.
\item [Task]�Piece of work assigned to a worker. Different type of tasks are described later. Every task assigns a single car to a single worker. 
\item [Administrator] Employee of PowerEni that is in charge of the administration of the system through the administration console.
\item [Drop off] The act of leaving the car inside a safe area. This ends the service provided to the driver and effectively make the driver pay
\item [Safe Area] An area in which the Driver can park the car and stop the Service. 
\item [RASD] Requirements Analysis and Specifications Document;
\item [DD] Design Document;
\item [API] Application Programming Interface; 
\end{description}


