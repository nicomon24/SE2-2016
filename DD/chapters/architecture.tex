\section{Overview}
The PowerEnjoy system is designed as a 3-tier application:
\begin{itemize}
\item \textbf{Data tier:} A PostgreSQL DB, deployed on AWS and replicated for availability and integrity. 
\item \textbf{Application tier:} A NodeJS application that provides APIs, deployed on AWS and replicated. The Express library is used to realise API endpoints. These APIs are exposed in a REST manner, responses are JSONs. The replication of the application grants fault tolerance and, using a load balancer, also availability.
\item \textbf{Presentation tier:} The presentation layer is realised with the 2 mobile applications (Driver and Worker), the web portal (Driver) and the Administrator application. The web portal is realised using EJS templating (server-side) and JQuery (client-side).
\end{itemize}
\includegraphics[width=13cm,keepaspectratio]{archi}
\newpage
\section{High level components}
The high level structure of the System can be divided into 6 components. The main component is the \textbf{core}, that receives requests from \textbf{clients}, either \textbf{drivers}, \textbf{workers} and \textbf{administrators}; depending on the client type, the component provides different interfaces. 
The core also communicates with the \textbf{DBMS} component, responsible of data persistence. 
The core is also responsible of tracking vehicle states, by polling them every 30 seconds with a state request (HTTP GET).
The core implements all the business logic: 
\begin{itemize}
\item Process drivers requests and allocates the requested vehicle 
\item Manages vehicles, track their state
\item Manages users (drivers signup, login)
\item Manages reports and automatic tasks then it allocates them to workers
\end{itemize}
All the communications by clients (Driver, Worker, Admin) are realised in a synchronous client-server architecture, where the core is the server. The communications with the DBMS are also client-server (the core is the client) and are realised using PostgreSQL protocol. The communications with the vehicles uses HTTP but the core is the client, while vehicles are servers.
The communication with the DBMS and the data modelling will be handled using an external NodeJS library, Sequelize.JS. In this way, not only we are simplifying our job, but also rely on a community tested library, i.e. more reliable.

\includegraphics[width=13cm,keepaspectratio]{maincompo}
\newpage
\subsection{Tracking vehicle state and scope}
Tracking vehicle states requires a more detailed explanation. Our objective is to provide a simple interface for the core to communicate with vehicles (both tracking and controlling), so that the core can perform simple GET requests to endpoints (e.g. /vehicle/vehicle-id) to know their state (e.g. position, occupants) or POST request to perform actions on vehicles (e.g. open/close doors). Every vehicle is equipped with a wireless communication system, and is connected to the internet (so it has a public and static IP address). The core associate every vehicle ID with its IP address (and store this address in the DB), so every time a request to a specific vehicle is made, the core resolves the IP address of the requested vehicle. For security reasons, vehicles accept request only from the core IP (that is static). Every vehicle implements a simple on-board server, that answers to the core GET/POST requests, i.e. providing the vehicle's state or performing actions.
Since our main objective is to design the management system, we will take as an assumption the implementation of the vehicle's on-board system. So, we could consider the Vehicle themselves as external components of the system. The team that developed the on-board system, provided us the following APIs:
\begin{itemize}
	\item \textbf{GET /state} returns the current state of the vehicle. The state is a JSON object containing \begin{itemize}
		\item Current position of the vehicle
		\item Current battery charge
		\item Maintenance report (if any issue is found)
		\end{itemize} 
	\item \textbf{GET /drive} returns the log of the last drive. This log includes \begin{itemize}
		\item Drive time
		\item Drive length
		\item Start and stop positions
		\item Number of occupants
		\end{itemize}
	\item \textbf{POST /unlock} unlocks the car and start logging the drive
	\item \textbf{POST /lock} locks the car and stop logging
	\item \textbf{POST /maintenance/[on-off]} enable or disable maintenance mode, where the car disables all is security systems. 
\end{itemize}
\newpage
\section {Core component}
Since the other components are quite simple and atomic, only the core component need some further analysis. The core has many functionalities that are divided in different components (for component diagram, see Paragraph 2.3.1):
\begin{itemize}
\item \textbf{Router} This component is responsible of receiving all the HTTP request to the APIs of the core and responding to them. It routes the received request from all the clients (driver, worker, admin) and from the vehicles to the specific component, so it provides all the interfaces with the external world. For every request, the router also communicates with the Authorization Manager to check if the calling user is authorized to perform certain actions. The Router always receives HTTP data as input and parses it. All the subsequent calls to subcomponents also contains the identifier of the user that has called it. The recognize this user, the Router calls a method (checkPermission) on the AuthorizationManager.
\item \textbf{Vehicle Manager} This component is responsible of tracking vehicle usage state (see state chart, paragraph 2.4.1) but also vehicle internal state (e.g. position). This states are saved to the DB to provide persistency but also analytics. It provides an interface to the Drive Controller and the Task Manager to get information about the car and to update their state. In order to be updated about the state of the vehicles the Vehicle Manager keeps polling them. The vehicle can also answer with a report of some issue, so the VehicleManager is also able to call TaskManager to generate a new report.
\item \textbf{Drive Controller} This component is responsible of processing Drivers requests (e.g. reservation, cancellation, termination). It communicates with the Vehicle Manager to update vehicle state. It also communicates with the Payment Manager to process Drivers payments. It is also responsible of computing the correct amount due at the end of a drive. Once the drive has ended when the user has pressed the button on the vehicle, the DriveManager requests to the VehicleManager the log of the drive, it computes the base amount to pay and applies discounts. A Node.JS implementation of the amount computing is provided in chapter 3.
\item \textbf{Payment Manager} It receives request to make a specific user pay a specific amount, process the payment and store it in the DB if the payment was successful. It is also responsible of verifying with the pre-authorization if the Driver can pay. 
\item \textbf{Task Manager} It is primarily responsible of managing task allocation to workers, but also of processing Drivers reports and Admins special tasks queries. It exposes an interface to the Vehicle Manager too in order to allow it to perform automatic maintenance tasks on vehicles. A Node.JS implementation of the worker choosing algorithm is provided in chapter 3.
\item \textbf{Authorization Manager} It manages most of the security of the system. It is responsible of Drivers and Workers signup and login. It is also responsible of verifying that every requests to the core are performed by an authorized user (e.g. Driver can't request Workers API). 
\end{itemize}
\subsection{Core Component Diagrams}
\includegraphics[width=13cm,keepaspectratio]{subcomponents12}
\includegraphics[width=13cm,keepaspectratio]{classborder}
\newpage
\section{Relevant State Charts}
To give a deeper explaination of the dynamic behaviour of the model we add here some State Chart Diagrams, describing the lifecycle of vehicles, reservations and tasks:
\subsection{Vehicle State Chart}
\includegraphics[width=13cm,keepaspectratio]{vehiclestate}
\subsection{Reservations State Chart}
\includegraphics[width=13cm,keepaspectratio]{reservationstate}
\subsection{Task State Chart}
\includegraphics[width=13cm,keepaspectratio]{taskstate}
\newpage
\section{Deployment View}
This section highlights the real deployment of the different components deepening the presentation already made in the previous paragraphs.
\includegraphics[width=13cm,keepaspectratio]{deployment}
\newpage
\section{Runtime View}
Below 2 UML sequence diagrams are provided, to show the dynamic behaviour during a complete drive and during a task assignment and completion.
\subsection{Drive Sequence Diagram}
\includegraphics[width=12cm,keepaspectratio]{drive}
\subsection{Task Assignment Sequence Diagram}
\includegraphics[width=13cm,keepaspectratio]{taskassignment}
\newpage
\section{Interfaces}
Here the interface of all the subcomponents of the system are provided. The 4 interfaces of the Router are the ones exposed to clients as in the figure of section 2.3.1.
\includegraphics[width=13cm,keepaspectratio]{interfaces}

