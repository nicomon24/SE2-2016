After a risk analysis, we discovered several possible issues that may occur during the development of the project. They are divided in Project Risks, Technical Risks and Legal and Financial Risks.

\subsection{Project Risks}
\begin{itemize}
\item Requirement Change: during the development of the project it could happen that our stakeholders change some requirement. This eventuality is completely unpredictable and the risk level is really high both for probability (it is not uncommon) and because it would require a partial restructuration of the System. This risk can be mitigated using a modular design and writing reusable code or letting the stakeholders be an active part of the project (i.e. using Agile methodology with frequent deliveries);
\item Lack of Experience: since it is the first time we commit in the development of a large scale system, some problems caused by inexperience could arise. For example, during the implementation phase, it could happen that one of our developers is not able to solve a specific issue, causing delays. This problem can be partially mitigated hiring flexible developers. Also team working could be fundamental in this scenario.
\end{itemize}
\subsection{Technical Risks}
\begin{itemize}
\item Security Problems: when the service becomes publicly accessible there is a chance that malicious attackers could try to steal data from our servers. Since we deal with personal information of Drivers and credit cards it is of prime importance that we put all our effort in vulnerability detection and fixing. Also architectural choices can play an important role in this scenario. 
\item Performance Problems: it is not easy to foresee the magnitude of workload of the System. If the application becomes very popular it could be that the computational power or the storing space available on our servers are not sufficient to handle the traffic. To overcome this issue we should use a scalable cloud computing service which is able to increase its power when needed.
\item Dependence on Third Party Software: the development of such a complex system implies a massive use of external services. Some of them could be currently under development. For example it could happen that Google releases a new version of Android OS with new APIs. This could cause delays in the project schedule because our components should be readapted. The bad news is that there are no countermeasures for issues of this type and it is also hard to foresee them.
\end{itemize}

\subsection{Legal and Business Risks}
\begin{itemize}
\item Legal Problems: since our cars are supposed to be driven trough public streets the chance of accidents or malfunctions is really high. In this case the System and the cars should be highly reliable because if something goes wrong the responsibility is ours. To overcome this issues we should plan an accurate maintenance schedule and more importantly contact a legal firm.
\item Competition: nowadays there are tons of car sharing services in every city. To persuade people to use ours service in place of another we should advertise it in a proper way, focusing on its main assets: electric cars, low cost, simple interface. 
\end{itemize}