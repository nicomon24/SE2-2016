In this Chapter, every step of the Code Inspection checklist is documented. 

\section{Naming Conventions}
\begin{enumerate}
\item [1] Variables names are meaningful 
\item [2] One-char variables are used only as exceptions. 
\item [3] Class names are mixed case with first uppercase letter.
\item [4] No interface is declared here
\item [5] All method names start with verbs and have mixed case
\item [6] All class variable have mixed with starting lower case char (no underscores either)
\item [7] This class has no constants
\end{enumerate}

\section{Indention}
\begin{enumerate}
\item [8] Four space are used to indent every time
\item [9] No tab are used in the file
\end{enumerate}

\section{Braces}
\begin{enumerate}
\item [10] All braces are in K\&R style.
\item [11] Every single statement block is braced correctly
\end{enumerate}

\section{File Organization}
\begin{enumerate}
\item [12] Section are well separated by spaces
\item [13] \textcolor{red}{Many lines exceed the 80 char limit even if not necessary.} Lines: 74, 86, 114, 116, 121, 198, 201, 228, 249, 268, 270, 278, 280 can be written more elegantly. Other lines exceed 80 char limit but can be accepted. 
\item [14] \textcolor{red}{Some lines also exceed 120 char limit.} Lines: 74, 228, 268, 270, 280
\end{enumerate}

\section{Wrapping lines}
\begin{enumerate}
\item [15] \textcolor{red}{Some operators are inserted after the line break.} Lines: 163, 164, 165, 166
\item [16] High-level breaks are used.
\item [17] \textcolor{red}{New statements are not aligned with the beginning of the previous line at the same level} Lines: 162, 163, 164, 165, 166
\end{enumerate}

\section{Comments}
\begin{enumerate}
\item [18] \textcolor{yellow}{Comments could explain more.} Some method's function could be better explained even if quite simple. 
\item [19] There is no commented out code.
\end{enumerate}

\section {Java Source Files} 
\begin{enumerate}
\item [20] This java contains only one Java Class.
\item [21] The public class is the first class of the file
\item [22] As in the Javadoc, this Class does not expose any external interface, only public variables to be read-only.
\item [23] The Javadoc is practically complete.
\end{enumerate}

\section {Package and Import Statements}
\begin{enumerate}
\item [24] The first non comment line is the package, all the imports follows. 
\end{enumerate}

\section{Class and Interface Declarations}
\begin{enumerate}
\item [25]�Declarations follow the order provided. 
\item [26]�Since methods perform very different tasks, it is difficult to find a correct order.
\item [27] Code is free of duplicates, long methods and big classes.
\end{enumerate}

\section {Initialization and Declarations}
\begin{enumerate}
\item [28]�All variables and class members are of the correct type. Moreover they have the right visibility.
\item [29] All variables are declared in the proper scope.
\item [30] When new objects are created the constructor is always called.
\item [31] All object references are initialised before use.
\item [32] Variables are initialized where they are declared, in rare cases (e.g. line 60 and 225) they depend on computation so they cannot be initialised before.
\item [33] Declarations always happen at the beginning of a block.
\end{enumerate}

\section {Method Calls}
\begin {enumerate}
\item [34] All methods are called properly with all parameters in the correct order.
\item [35] The correct method is always being called.
\item [36] Method returned values are used properly.
\end{enumerate}
\section {Arrays}
\begin{enumerate}
\item [37] There are no off-by-one errors (for Lists and ArrayLists the safe method .add() is always used)
\item [38] All array indexes have been prevented from going out-of-bounds.
\item [39] Constructors are always called when a new array item is desired.
\end{enumerate}
\section {Object Comparisons}
\begin{enumerate}
\item [40] \textcolor{red}{Some objects are compared with == instead of equals().} Lines: 162, 163, 164, 167, 171, 174.
\end{enumerate}
\section {Output Format}
\begin{enumerate}
\item [41] The displayed output is free of spelling and grammatical errors.
\item [42] The error messages are comprehensive but they \textcolor{red}{do not provide} any guidance on how to correct the problem.
\item [43] The output is correctly formatted.
\end{enumerate}
\section {Computation, Comparisons, Assignments}
\begin{enumerate}
\item [44] The file is a simple configuration class, hence no expensive computation is carried out. So the implementation of course avoids "brutish programming".
\item [45] The order of computation/evaluation is correct. there are no computations needing parenthesis.
\item [46] Computations are really simple and done one by one, hence there is no need to use parenthesis.
\item [47] There are no denominators in this class.
\item [48] There are only simple integer additions and the two addends are always integers. Hence the implementation of course avoids truncation/rounding.
\item [49] Comparisons and Boolean operations are always correct.
\item [50] Try-Catch exceptions are fine and the error condition is always legitimate.
\item [51] The code is free of implicit type conversions.
\end{enumerate}
\section {Exceptions}
\begin{enumerate}
\item [52] All the relevant exceptions are caught.
\item [53] Appropriate actions are taken for each catch block.
\end{enumerate}
\section {Flow of Control}
\begin{enumerate}
\item [54] In the switch statement found in line 184, the last case (line 191) \textcolor{red}{does not contain break or return.}
\item [55] The switch statement found in line 184 \textcolor{red}{does not have a default branch.}
\item [56] All loops are correctly formed, but the while loop (line 246) \textcolor{yellow}{could be written in a more elegant way.}
\end{enumerate}
\section {Files}
\begin{enumerate}
\item [57] All files are properly declared and opened.
\item [58] All files are closed properly, even in the case of an error.
\item [59] All EOF conditions are detected and handled correctly.
\item [60] All file exceptions are caught and dealt with.
\end{enumerate}


