\section{Entry criteria}
Integration Testing can start as long as the entry criteria stated below are met. First of all, the RASD and the DD documents must have been completed and accepted, since we need a complete view of the problem and the design of the system. \newline
Also, integration should start only when the estimated percentage of completion of the various components met this requirement
\begin{itemize}
\item TODO
\end{itemize}
This percentage describe only the entry criteria for the integration testing phase, not the actual integration test of the component (obviously possible only when the component is almost complete). 

\section{Elements to be integrated}
In the DD, the structure of the system is clearly divided into high-level components, e.g. the Core and Clients, and lower-level component, i.e. the subcomponents of the Core. So, the integration phase will be performed at different level of abstraction. Given that the lower-level components compose the essential high-level component of the system (the Core), we will first integrate the lower-level and then proceed to higher levels. \newline
The first critical component of the system is the Data Access Layer, that is implemented through an external Node.JS library (Sequelize, DD v1.1). For this reason, all the CRUD operations (Create, Read, Update, Delete) on the DB are considered as already tested. The usage of these operations inside components are consequentially already tested in Unit Testing.
The lower-level components to be tested in the first phase are: \textbf{Vehicle Manager, Drive Manager, Payment Manager, Router, Authorization Manager and Task Manager}.
\newline
TODO: describe high level components

\section{Integration Testing Strategy}
We are going to use mainly a bottom-up approach during the integration testing of lower-level components. So, we will start integrating the components that does not depend on other components or depend on already developed components. Since we have many simple components that are very independent (Vehicle Manager, Payment Manager, Authorization Manager), this approach gives us the advantage to begin the testing phase earlier and start to integrate as soon as components are ready and functional. 
The second phase will follow a critical-first approach, since the components here are only dependent to the Core. So, the order will reflect the risk represented by the incorrect behaviour of the component.

\section{Sequence of Integration}

\subsection{Software integration}

\subsection{Subsystem integration}

\section{Test items}
The items are the integrations of the components previously described in [Design Document, DD, Paragraph 2.X]. 

++++++++++++++++++++++++++++++++
QUI SCRIVO PER PRENDERE APPUNTI

Prima di tutto prendiamo il Unit Testing come gi� fatto (questo � l'integration)
Dovremmo fare un bottom-up
Riprendendo il grafico al paragrafo 2.7 la sequenza di integrazione potrebbe essere 

Va modificato l'interfaces, manca la connessione da Router Vehicle a Drive Manager per notificare la fine di una drive, e tra router admin e Vehicle per le statistiche

I1: Drive Manager -> Vehicle Manager
Predisporre una stub dei pagamenti
Driver al Drive Manager per testare: reservation, cancellation, start and stop

I2: Drive Manager -> Payment Manager
Testare tramite driver al drive manager: limite di un'ora con pagamento, fine di una drive con pagamento

I3: Router Vehicle -> Drive Manager
Testare la end drive dal veicolo, simulare con driver il veicolo stesso

I4: Task Manager -> Vehicle Manager
Testare tramite driver la creazione di un report e di un task diretto, l'update di un task, il cambio di stato del worker

I5: Router Vehicle -> Task Manager, Authorization
Testare la malfunction di un veicolo, driver veicolo 

I6: Router Driver -> Task Manager, Drive Manager, Authorization
Testare tutte le funzioni lato driver, simulando l'applicazione stessa

I7: Router Worker -> Task Manager, Authorization
Testare tutte le funzioni lato Worker

I8: Router Admin -> Task Manager, Authorization
Testare tutte le funzioni lato Admin
