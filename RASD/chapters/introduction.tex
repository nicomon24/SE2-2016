\section{Description of the problem}
We are to project and develop a car-sharing system, PowerEnJoy. This service only relies on electric cars, so more attention is paid to the driver behaviour to reduce infrastructure cost, minimizing the number of necessary charging stations. The system has to interact primarily with the users that want to rent the car, but also with "\textbf{workers}", employees of the company PowerEni, that are responsible for the management of cars themselves. All the cars in the system are connected to the internet through a mobile data receiver and are equipped with several sensor (GPS, speedometer). The system must also provide an \textbf{administration console}, accessible by few users inside PowerEni company. This console provides administration functionalities (e.g. registering new workers, log the state of all the vehicles). The main users of the system are\\
\begin{description}
\item [User] Users interact with the system primarily through a mobile application, given that users usually rent cars while they are on the move. The application allows users, or "\textbf{drivers}", to reserve a car by presenting a list or map of all the near-by available cars (using the user GPS position). The application allows the user to effectively use the car (find where it is parked, open the doors, find useful informations, e.g. near-by charging stations). The system also provides a web portal, where the user can manage all his informations (payment settings, license, view payment). Users can access the system only through a registration, that can be done both on the web portal and on the mobile application. During the registration, the user need to provide payment informations and a valid driving license. 
\item [Workers] Workers interact with the system through a specific mobile application, released only to them. Workers are paid by the company to execute "tasks", that include:
\begin{itemize}
\item Recharging on-site a car that was not left in a charging station
\item Move cars to more popular places to maximize usage
\item Pick up a car in the case of an accident or normal maintenance
\end{itemize}

\item [Administrator] Interact with the system through a web application. Given that this is a main asset for the company, because it can effectively block the system, the security of this web application should be very high: possibly a web application accessible only inside PowerEni network with a 2 factor authentication. Once logged in, the administrator can perform these operations
\begin{itemize}
\item Register a new worker newly employed in the system, generating a username and a password (and unregister them)
\item View the status of a single car (e.g. position, battery charge) and request exceptional tasks on it (not the ones generated automatically by the system)
\end{itemize}

\end{description}

\section{Goals}
\textbf{Drivers goals}
\begin{itemize}
\item \textbf{[GOAL1]} Log in the system.
\item \textbf{[GOAL2]} Allow drivers to reserve a car up to one hour before they pick it up
\item \textbf{[GOAL3]} Allow drivers to open the reserved car
\item \textbf{[GOAL4]} Allow drivers to pay correctly for the service
\item \textbf{[GOAL5]} Allow drivers to cancel a reservation
\end{itemize}

\textbf{Workers goals}
\begin{itemize}
\item \textbf{[GOAL50]} Log in the system
\item \textbf{[GOAL51]} Dispatch tasks to workers
\item \textbf{[GOAL52]} Allow workers to get informations about an assigned task
\item \textbf{[GOAL53]} Allow workers to open and control the assigned car
\item \textbf{[GOAL54]} Allow workers to keep track of the state of the task (assigned, in progress, done)
\end{itemize}

\textbf{Administrator goals}
\begin{itemize}
\item \textbf{[GOAL100]} Log in the administration console 
\item \textbf{[GOAL101]} Allow to register new workers
\item \textbf{[GOAL102]} Allow to view the status of a specific vehicle
\item \textbf{[GOAL103]} Allow to request an exceptional task on a specific vehicle
\end{itemize}

\section{Domain assumptions}
\begin{itemize}
\item Every car is equipped with some wireless communication system to connect it the the internet.
\item Every car is equipped with GPS, speedometer, passenger presence sensors, engine state sensors, data is sent remotely to the system through the internet.
\item Every car is equipped with a remotely controllable locking system.
\item Every car is legally usable on public street (insurance, taxes, maintenance)
\item Every time the user reserves a car, the payment is pre-authorized by an amount equal to the rent of a day
\item Workers are registered in the system by the company itself. The worker application is only distributed to them and not on public stores.
\item Every user can reserve a car at a time.
\item Every worker must have only one task assigned at a time.
\item Drivers always respect the driving rules and are able to drive.
\item Driver-provided license informations are assumed to be truthful
\item Safe areas are considered already defined by the management system and deployed to the vehicle on-board system.
\end{itemize}

\section{Glossary}
\begin{description}
\item [Driver] Client of the system, i.e. the one that uses the service, reserves and drives. Every Driver must provide personal informations (name, surname, email, birthdate, a valid license and payment information.
\item [Drive] A single usage of the service, starts when the user turn on the engine, stops when the car is locked inside a Safe Area;
\item [Vehicle] An electric car that is used to provide the service.
\item [Blocked driver] A driver that has an invalid license or invalid payment coordinates
\item [Worker] Employee of the company, perform physical actions (moving, charging etc) on vehicles
\item [Report] Vehicle issue reported by the Driver during a Drive. Will later be assigned to a Worker.
\item [Task]�Piece of work assigned to a worker. Different type of tasks are described later. Every task assigns a single Vehicle to a single worker. 
\item [Administrator] Employee of PowerEni that is in charge of the administration of the system through the administration console.
\item [Drop off] The act of leaving the Vehicle inside a safe area. This ends the service provided to the driver and effectively make the driver pay
\item [Safe Area] An area in which the Driver can park the Vehicle and stop the Service. 
\end{description}

\section{Stakeholders}
The stakeholder is the PowerEni company, who needs the system to provide the service itself.  
\section{References}
\begin{itemize}
\item IEEE Std 830-1998 IEEE Recommended Practice for Software Requirements Specifications.
\item Example documents:
\begin{itemize}
\item RASD sample from Oct. 20 lecture.pdf
\end{itemize}
\end{itemize}
